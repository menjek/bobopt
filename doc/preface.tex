\chapter{Introduction}
The Bobox project is a task-based framework for parallel computing. In such a framework, short-running tasks cause bigger CPU consumption by the framework itself and long-running tasks can inhibit the parallel execution. A static code analysis can be used to detect and eliminate such execution paths in the user code. Since the core and interface language of the Bobox framework is C++, static code analysis becomes more difficult in proportion to the lack of tools. This has been the biggest pitfall for any static analysis of C++ code. However, it has become less marginal with a growing support for tooling in the Clang compiler front-end~\cite{clang}. Clang exposes C++ code as a user-friendly \emph{Abstract Syntax Tree (AST)}~\cite{ast} structure.

\section{Goals}
Apart from this text, the main asset of this thesis is a tool for optimizing code using the Bobox framework. By analysing AST, the tool is able to diagnose and potentially transform user code to eliminate both short and long execution paths. There is an implementation of the two different patterns of unoptimized usage in the context of this thesis. However, the tool is designed to be easily extensible with new optimization methods. Both implemented optimization methods are able to inject new code to give the Bobox internal facilities information about the user code structure. The tool is implemented using the Clang tooling interface~\cite{clang-documentation}. Therefore, it inherits all the Clang limitations, such as its platform support.

\section{Structure of the thesis}
The second chapter begins with a brief description of the Bobox framework, in order to familiarize a reader with its underlying mechanisms. The reader should understand why the user code can be optimized using the transformations mentioned later in chapters related to specific optimization methods.

The third chapter describes the problems of static code analysis first of all in general, and then specifically for the C++ language since it is the core and interface language of the Bobox framework. The chapter also describes the possible approaches to a static code analysis of C++ code, their advantages and disadvantages, the internal implementation, and the user interface and limitations. The last section in this chapter addresses related work.

Because the chosen approach to implement the optimizer tool was the Clang tooling interface, Chapter~\ref{chapter-clang} is dedicated to its detailed description. Tools implemented on top of the Clang front-end are equipped with an abstract syntax tree representation of code. One section describes the design of this data structure, and the possibilities of its traversal. Possibilities of source-to-source transformations are described in the following section. The Clang front-end itself provides multiple interfaces for an implementation of static code analysis tools.

The next two chapters introduce implemented optimization methods. Each chapter contains sections with information related to the Bobox framework, a detailed description of the algorithms used to detect an unoptimized usage of the framework and other details of the implementation.

Chapter~\ref{chapter-design} offers a high-level look at the tool design and some details of optimizer implementation. The following chapter contains achieved results of the described scenarios. The last chapter discusses conclusions and the future work.