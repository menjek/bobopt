\chapter{Introduction}
The Bobox project is a task-based framework for parallel computing. In such framework, short running tasks cause bigger CPU consumption by the framework itself, long running tasks can inhibit the parallel execution. A static code analysis can be used to detect and eliminate such execution paths in user code. Since the core and interface language of the Bobox framework is C++, static code analysis becomes accordingly difficult to lack of tools. This has been the biggest pitfall for any static analysis of C++ code. However, it has become less marginal with a growing support for tooling in the Clang compiler front-end \cite{clang}. Clang exposes C++ code as a user-friendly \emph{Abstract Syntax Tree (AST)} \cite{ast} structure.

\section{Goals}
Apart from this text, the main asset of this thesis is a tool for optimization of code using the Bobox framework. By analysing AST, the tool is able to diagnose and potentially transform user code to eliminate short and long execution paths. There is an implementation of two different patterns of unoptimized usage in the context of this thesis, but the tool is designed to be easily extensible with new optimization methods. Both implemented optimization methods are able to inject new code to hint Bobox internal facilities about user code structure. The tool is able to perform source-to-source transformations so it can be used in a build process. The tool is implemented using Clang tooling API \cite{clang-documentation}. Therefore, it inherits all Clang limitations such as its platform support.

\section{Structure of the thesis}
The second chapter begins with a brief description of the Bobox framework to familiarize a reader with its underlying mechanisms. A reader should understand why user code can be optimized using transformations mentioned later in chapters related to specific optimization methods.

The third chapter describes problems of a static code analysis firstly in general, then specifically for the C++ language since it is the core and interface language of the Bobox framework. The chapter mentions possible approaches for a static code analysis of C++ code, their advantages and disadvantages, internal implementation, user interface and limitations. The last section addresses related work.

Because the chosen approach to implement the optimizer tool was the Clang tooling interface, chapter \ref{chapter-clang} is dedicated to its detailed description. Tools implemented on top of the Clang front-end are provided with the abstract syntax tree representation of code. One section describes design of this data structure and possibilities of its traversal. Possibilities of source-to-source transformations are described in the following section. The Clang front-end itself provides multiple interfaces for an implementation of static code analysis tools.

Next two chapters introduce implemented optimization methods. Each chapter contains sections with information related to the Bobox framework, detailed description of algorithms used to detect unoptimized usage of the framework and other details of the implementation.

Chapter \ref{chapter-design} addresses a high-level look on a tool design and some details of the optimizer implementation.