\chapter{Bobox}
Nowadays, increasing performance comes with an increasing number of computational units due the attainment of the physical limits of current technologies. Parallel programming becomes more and more important in the development of performance expensive software to use all the silicon hardware provides. The thread-based approach to achieve parallelism creates a lot of complexity for a programmer to maintain, and it is also not well scalable. It becomes important to abstract the underlying parallel environment to a programmer. The Bobox framework~\cite{bobox-4, bobox} addresses this issue by providing an interface for task-based parallel programming. Such an approach relieves the programmer of handling thread-based programming problems such as synchronization, most technical details (e.g., cache hierarchy, CPU architecture) and communication. Apart from that, task-based programming also allows for better hardware utilization.

According to Bednárek et al.~\cite{bobox-1}, the Bobox framework is \textit{"more useful for data processing scenarios, like database query evaluation or stream processing"}. Without further details, under the hood, the framework has been equipped with a fixed number of worker threads, each one with an own scheduler, and two task queues for every computational unit for a better utilization of CPU caches, using the task stealing mechanism. Communication between tasks uses a column-based data model, the most significant implementation detail that favours data processing problems. Each task has zero or more inputs and zero or more outputs. A single output can be connected to zero or more inputs. A task is scheduled to be executed when it has an unprocessed input.

The Bobox framework provides a C++ library as the interface to its runtime environment. Task granularity is represented by classes derived from the Bobox base class for a task. This base class is called a \emph{box}.

\section{Design and terminology}
\label{bobox-terminology}
The runtime environment handles the implementation details of the task-based parallel environment such as scheduling and the parallel execution of tasks, data transport and control flow. A programmer uses a declarative way to provide the environment with a \emph{model} which defines the way individual tasks are interconnected. The model is used to create a \emph{model instance} which is the base for creating a \emph{user request}. The user request contains only very little additional information compared to the model instance.

After a programmer provides the environment with a user request, he no longer has control over its execution. The framework provides only information about whether it has finished executing the request. The provided user request is divided into individual tasks. When a task is ready to be executed, it is added to the task pool. A worker thread then retrieves the task from the task pool and invokes it.

The basic element of model instance is an element representing a task, called a box. In every model instance, there is a special box called an \emph{initialization box}. This box is responsible for the creation of the initialization input data of all the other boxes in the model. The framework executes this task at the beginning of a request evaluation, and its only goal is to send data to its only output.

Data is sent using an \emph{envelope}, a column-based data structure. An empty envelope is a special type of envelope called a \emph{poisoned pill}. When a box receives a poisoned pill in its input, there will be no more data sent to this input. All the paths of model instances are required to end in another special type of box called a \emph{termination box}. When this box receives a poisoned pill, the execution is finished and the pipeline is deallocated.

\section{Boxes}
\label{bobox-boxes}
Boxes, as the representation of Bobox framework tasks, are executed in three steps.

\begin{enumerate}
\item The first step is the \emph{prologue}, when the box creates a snapshot of its inputs and stores this snapshot in member variables so the user code can access it. The prologue communicates with the runtime environment, and synchronization is needed.
\item The second step called \emph{action} is the main place for a user code execution. User code can communicate with the runtime environment using only specific member functions, e.g., it can send an envelope to its output. This approach creates a transparent parallel environment for programmers, relieving them of issues related to the parallel execution.
\item The last step is the \emph{epilogue}. The step handles the scheduling of the next task based on two criteria. A task is scheduled again,

\begin{enumerate}
\item if it has got an unprocessed input and it has processed some input in the action step.
\item if it requested to be scheduled again.
\end{enumerate}

There is a reason why box is not scheduled again if it has got unprocessed input and it has not processed any input during the execution. It will most likely wait for another input, e.g., the join operation in a database when a task is not executed until it has received data from both inputs. The option to explicitly request another scheduling is there for cases when a single box creates a large output. A task should not run for a long period of time. It can create a large output, and thus congest internal buffers used for communication. A task running for a long time on a single worker thread can also create a bottleneck for a parallel execution when many other tasks wait for the input from this task.

\end{enumerate}

Boxes are main objects of the interest for optimization, because they are the main location for the user code. Based on the static analysis of the action step, additional code can be injected to provide Bobox internal facilities with information about the task.

\section{Usage}
\label{bobox-usage}
To implement a Bobox task, a programmer has to inherit from the \code{basic\_box} base class. The action step is represented by one of the virtual member functions from Listing~\ref{bobox-action-step}. A programmer is expected to override one of them. The synchronous version is called when all prefetched envelopes are available. The asynchronous version is called when the envelope on the input which is being \textit{listened} is available.

\begin{lstlisting}[caption={The code representations of the box action step.},label={bobox-action-step}]
virtual void sync_body();
virtual bool async_body(inarc_index_type inarc);
\end{lstlisting}

A programmer can also associate names to particular inputs and outputs using helper macros from the Bobox library, see Listing~\ref{bobox-macros}. The code is easier to comprehend and maintain when box inputs or outputs are referred to by names instead of using indexes.

\begin{lstlisting}[caption={The helper macros for mapping of names to inputs and outputs.},label={bobox-macros}]
#define BOBOX_BOX_INPUTS_LIST(...)
#define BOBOX_BOX_OUTPUTS_LIST(...)
\end{lstlisting}

The implementation of a task is straightforward for a C++ programmer using only C++ core language features. Unfortunately, C++ syntax is not convenient for expressing a definition of the whole execution model. The language developed to such purpose is called \emph{Bobolang}~\cite{bobolang}. Listing~\ref{bobox-bobolang} shows an example of a model definition in the Bobolang language.

\begin{lstlisting}[caption={An example of the Bobolang usage.}, label={bobox-bobolang}]
model main<()><()> {
    bobox::broadcast<()><(),()> broadcast;
    Source<()><(int)> source1(odd=true), source2(odd=false);
    Merge <(int),(int)><(int)> merge;
    Sink <(int)><()> sink;
	
    input -> broadcast;
    broadcast[0] -> source1;
    broadcast[1] -> source2;
    source1 -> [left]merge;
    source2 -> [right]merge;
    merge -> sink -> output;
}
\end{lstlisting}

\section{Cooperative scheduling}
Due to the character of the framework scheduler, the user code directly affects the scheduling of tasks and thus the overall performance of an execution. The framework provides ways of manipulating the task scheduling. For example, a task can give up its execution before it finishes naturally. A task can also inform the scheduler that it should not be executed before all the input data is available. Based on the static code analysis of the user code, the optimizer tool can inject function calls into the user code to affect framework scheduling.